\documentclass{beamer}
\usetheme{Szeged}
\usecolortheme{beaver}

\usepackage[utf8]{inputenc}
\usepackage[russian]{babel}
\usepackage{amsmath}
\usepackage{fancyhdr}

\title{Научно-исследовательская практика}
\subtitle{Основы информационной безопасности}
\author{Гринчуков Владимир Владиславович, 2Кб}
\institute{БФУ им. И. Канта}
\date{\today}

\begin{document}
\begin{frame}
    \titlepage
\end{frame}

\begin{frame}
\frametitle{Описание курсовой работы}
\begin{center}
    \textbf{Тема курсовой работы:}\\
    «Угрозы и защита интернета вещей»
\end{center}
\emph{Задача курсовой:}\\
Рассмотреть текущее состояние безопасности в сфере IoT, обозначить возможные проблемы и уязвимости, привести примеры успешных атак, а также составить возможный план действий для обеспечения современных устройств наиболее защищенными системами.\\
\emph{Методы решения:}\\
Изучение темы на основе данных из свободных источников, обобщение результатов открытых исследований.
\end{frame}
\begin{frame}
\frametitle{Результаты курсовой работы}
В курсовой были указаны слабые стороны безопасности умных устройств, приведены примеры известных атак и указаны те уязвимости, которые эти атаки эксплуатировали для достижения цели. В заключительной части описаны те методы и средства, которые, по мнению специалистов, помогут производителям гаджетов и разработчикам программного обеспечения для них создавать наиболее защищенные системы, чтобы избежать нарушений конфиденциальности или неправомерного использования в будущем.
\end{frame}
\end{document}