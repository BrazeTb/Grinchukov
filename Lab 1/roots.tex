\documentclass[11pt]{article}

\usepackage[utf8]{inputenc}
\usepackage[russian]{babel}
\usepackage{amsmath}
\usepackage{fancyhdr}
\usepackage{amsmath,amssymb,amsthm}
\usepackage{algorithm}
\usepackage[noend]{algpseudocode} 

\pagestyle{fancy}

\fancypagestyle{first}{
\setcounter{page}{156}
\renewcommand{\headrulewidth}{0pt} 
}
\fancypagestyle{fancyEven}{
\fancyhf{}
\renewcommand{\headrulewidth}{0pt} 
\lhead{\textbf{\thepage}}
\chead{\textbf{Примитивные корни и индексы}}
\rhead{\textbf{Глава 8}}
}
\fancypagestyle{fancyOdd}{
\fancyhf{}
\renewcommand{\headrulewidth}{0pt} 
\lhead{\textbf{Часть 8-1}}
\chead{\textbf{Порядок числа по модулю $n$}}
\rhead{\textbf{\thepage}}
}
\newcommand*\PROB\Pr 
\DeclareMathOperator*{\EXPECT}{\mathbb{E}}

% Sets, Rngs, ets 
\newcommand{\N}{{{\mathbb N}}}
\newcommand{\Z}{{{\mathbb Z}}}
\newcommand{\R}{{{\mathbb R}}}
\newcommand{\Zp}{\ints_p} % Integers modulo p
\newcommand{\Zq}{\ints_q} % Integers modulo q
\newcommand{\Zn}{\ints_N} % Integers modulo N

% Landau 
\newcommand{\bigO}{\mathcal{O}}
\newcommand*{\OLandau}{\bigO}
\newcommand*{\WLandau}{\Omega}
\newcommand*{\xOLandau}{\widetilde{\OLandau}}
\newcommand*{\xWLandau}{\widetilde{\WLandau}}
\newcommand*{\TLandau}{\Theta}
\newcommand*{\xTLandau}{\widetilde{\TLandau}}
\newcommand{\smallo}{o} %technically, an omicron
\newcommand{\softO}{\widetilde{\bigO}}
\newcommand{\wLandau}{\omega}
\newcommand{\negl}{\mathrm{negl}} 

% Misc
\newcommand{\eps}{\varepsilon}
\newcommand{\inprod}[1]{\left\langle #1 \right\rangle}

\newcommand{\handout}[5]{
  \noindent
  \begin{center}
  \framebox{
    \vbox{
      \hbox to 5.78in { {\bf Научно-исследовательская практика} \hfill #2 }
      \vspace{4mm}
      \hbox to 5.78in { {\Large \hfill #5  \hfill} }
      \vspace{2mm}
      \hbox to 5.78in { {\em #3 \hfill #4} }
    }
  }
  \end{center}
  \vspace*{4mm}
}

\newcommand{\lecture}[4]{\handout{#1}{#2}{#3}{Scribe: #4}{Примитивные корни и индексы #1}}

\newtheorem{theorem}{Теорема}
\newtheorem{lemma}{Лемма}
\newtheorem{definition}{Определение}
\newtheorem{corollary}{Следствие}
\newtheorem{fact}{Факт}

% 1-inch margins
\topmargin 0pt
\advance \topmargin by -\headheight
\advance \topmargin by -\headsep
\textheight 8.9in
\oddsidemargin 0pt
\evensidemargin \oddsidemargin
\marginparwidth 0.5in
\textwidth 6.5in

\parindent 0in
\parskip 1.5ex

\begin{document}
\thispagestyle{empty}
\lecture{}{Лето 2020}{}{Владимир Гринчуков}


	
\begin{Large}\textbf{8.1 Порядок числа по модулю n} \end{Large}\\
\thispagestyle{first}
Из теоремы Эйлера нам известно, что $a^{\phi(n)}\equiv 1\pmod{n}$, когда $\text{НОД}(a,n)=1$. Тем не менее часто существуют степени $a$, меньшие, чем $a^{\phi(n)}$, но сравнимые с $1$ по модулю $n$. Отсюда следует определение:
\begin{definition}
	Пусть $n>1$ и $\text{НОД}(a,n)=1$. \emph {Порядком $a$ по модулю $n$} (в более старой терминологии: \emph{показатель степени, к которому принадлежит $a$ по модулю $n$}) называется наименьшее положительное целое число $k$ такое, что $a^k\equiv 1\pmod{n}$.
\end{definition}

Последовательно рассмотрим степени 2 по модулю 7. Для этого модуля получаем сравнения $$2^1\equiv2,\; 2^2\equiv4,\; 2^3\equiv1,\; 2^4\equiv2,\; 2^5\equiv4,\; 2^6\equiv1,\; ...,$$ из которых следует, что целое число 2 имеет по модулю 7 порядок равный 3.\\

Заметим, что если два целых числа сравнимы по модулю $n$, то они имеют один и тот же порядок по этому модулю. Так как $a\equiv b\pmod{n}$ и $a^k\equiv 1\pmod{n}$, из Теоремы 4-2 следует, что $a^k\equiv b^k\pmod{n}$, откуда $b^k\equiv 1\pmod{n}$.

Необходимо подчеркнуть, что наше определение порядка по модулю $n$ учитывает только такие числа $a$, для которых $\text{НОД}(a,n)=1$. Действительно, если $\text{НОД}(a,n)>1$, то мы из Теоремы 4-7 знаем, что линейное сравнение $ax\equiv1\pmod{n}$ не имеет решений; таким образом отношение
\begin{align}
a^k\equiv1 \pmod{n}\tag*{$k\geq{1}$}
\end{align}
не может иметь силы, так как это означало бы, что $x=a^{k-1}$ -- решение сравнения $ax\equiv 1\pmod{n}$. Таким образом, если упоминается порядок порядок $a$ по модулю $n$, предполагается, что $\text{НОД}(a,n)=1$, даже если явно это не указано.

В примере, данном выше, $2^k\equiv1\pmod{7}$, когда $k$ кратно 3, порядку числа 2 по модулю 7. Наша первая теорема показывает, что в общем случае это типичная ситуация.
\begin{theorem}
	\it Пусть целое число $a$ имеет по модулю $n$ порядок $k$. Тогда $a^h\equiv1\pmod{n}$ тогда и только тогда, когда $k|h$; в частности $k|\phi(n)$
\end{theorem}
\newpage
\thispagestyle{fancyOdd}
\begin{proof}
	Предположим для начала, что $k|h$, то есть $h=jk$ для некоторого целого $j$. Так как $a^k\equiv1\pmod{n}$, Теорема 4-2 гласит, что $(a^k)^j\equiv1^j\pmod{n}$ или $a^h\equiv1\pmod{n}$.
	
	Напротив, пусть $h$ -- любое положительное целое число, удовлетворяющее условию $a^h\equiv1\pmod{n}$. Согласно алгоритму деления существуют такие числа $q$ и $r$, что $h=qk+r$, где $0\leq{r}<k$. Следовательно $$a^h=a^{qk+r}=(a^k)^qa^r.$$	
	По предположению $a^h\equiv1\pmod{n}$ и $a^k\equiv1\pmod{n}$, откуда $a^r\equiv1\pmod{n}$. Так как $0\leq{r}<k$, приходим к тому, что $r=0$; другими словами выбор $k$ как наименеьшего положительного целого числа такого, что $a^k\equiv1\pmod{n}$ вызывает противоречие. Таким образом $h=qk$ и $k|h$.
\end{proof}

Теорема 8-1 упрощает поиск порядка целого числа $a$ по модулю $n$: вместо того, чтобы рассматривать все степени $a$, можно сократить их список только до  делителей $\phi(n)$. В качестве примера найдем порядок числа 2 по модулю 13. Так как $\phi(13)=12$, порядок нужно искать среди чисел 1, 2, 3, 4, 6, 12. Из сравнений $$2^2\equiv4\;,2^3\equiv8\;2^4\equiv3\;2^6\equiv12,\;2^{12}\equiv1\pmod{13}$$ видно, что 2 имеет по модулю 13 порядок, равный 12.

Для произвольно выбранного делителя $d$ числа $\phi(n)$ не всегда верно, что существует целое число $a$, порядок по модулю $n$ которого равен $d$. Например, $n=12$. $\phi(n)=4$, однако не существует целых чисел порядка 4 по модулю 12; действительно, $$1^2\equiv5^2\equiv7^2\equiv11^2\equiv1\pmod{12}$$ и единственные возможные порядки -- это 1 или 2.

Вот другое фундаментальное утверждение о порядке целого числа.
\begin{theorem}
	\it Если $a$ имеет порядок $k$ по модулю $n$, тогда $a^i\equiv a^j\pmod{n}$ тогда и только тогда, когда $i\equiv j\pmod{k}$.
\end{theorem}
\begin{proof}
	Предположим для начала, что $a^i\equiv a^j\pmod{n}$, где $i\geq{j}$. Так как $a$ взаимно прост с $n$, мы можем сократить степень $a$ и получить $a^{i-j}\equiv1\pmod{n}$. В соответствии с Теоремой 8-1, последнее сравнение справедливо только если $k|i-j$, а это значит, что $i\equiv j\pmod{k}$.
	
	Напротив, пусть $i\equiv j\pmod{k}$. Тогда $i=j+qk$ для некоторого целого $q$. По определению $k$, $a^k\equiv1\pmod{n}$, так что $$a^i\equiv a^{j+qk}\equiv a^j(a^k)^q \equiv a^j\pmod{n}$$
	Что и требовалось доказать.
\end{proof}
\begin{corollary}
	\it Если $a$ имеет порядок $k$ по модулю $n$, тогда $a,\;a^2,\;...,\;a^k$ не сравнимы по модулю $n$.
\end{corollary}
\newpage
\thispagestyle{fancyEven}
\begin{proof}
	Если $a^i\equiv a^j\pmod{n}$ для $1\leq i\leq j\leq k$, тогда теорема гарантирует, что $i\equiv j\pmod{k}$, однако это возможно только в том случае, если $i=j$
\end{proof}

Может возникнуть справедливый вопрос: возможно ли выразить порядок степеней $a$ через через порядок самого $a$?
\begin{theorem}
	\it Если целое число $a$ имеет порядок $k$ по модулю $n$ и $h>0$, тогда $a^h$ имеет порядок $k/\text{НОД}(h,k)$ по модулю $n$.
\end{theorem}
\begin{proof}
    Пусть $d=\text{НОД}(h,k)$. Тогда можно написать $h=h_1d$ и $k=k_1d$, где $\text{НОД}(h_1,k_1)=1$. Очевидно, что $$(a^h)^{k_1}=(a^{h_1d})^{k/d}=(a^k)^{h_1}\equiv1\pmod{n}.$$
    Если предполагается, что $a^h$ имеет по модулю $n$ порядок $r$, то по Теореме 8-1 $r|k_1$. С другой стороны, порядок $a$ по модулю $n$ равен $k$, следовательно сравнение $$a^{hr}\equiv(a^h)^r\equiv1\pmod{n}$$ указывает, что $k|hr$; другими словами, $k_1d|h_1dr$ или $k_1|h_1r$. Но $\text{НОД}(k_1,h_1)=1$, а значит, что $k_1|r$. Это отношение делимости, с учетом ранее сказанного, дает
    \begin{center}
    $r=k_1=k/d=k/d=k/\text{НОД}(h,k)$,
    \end{center}
    что доказывает теорему.
\end{proof}

Последняя теорема имеет следствие, доказательно которого может быть предоставлено читателем.

\begin{corollary}
	\it Пусть $a$ имеет порядок $k$ по модулю $n$. Тогда $a^h$ также имеет порядок $k$ тогда и только тогда, когда $\text{НОД}(h,k)=1$.
\end{corollary}

Давайте посмотрим, как все это работает, на конкретном случае.\\

\textbf{Пример 8-2}

В следующей таблице представлены порядки положительных целых чисел, меньших 13, по модулю 13.\\

\begin{tabular}{c|c c c c c c c c c c c c}
    число &1 &2 &3 &4 &5 &6 &7 &8 &9 &10 &11 &12\\
    \hline
    порядок &1 &12 &3 &6 &4 &12 &12 &4 &3 &6 &12 &2
\end{tabular}\\\\
\newpage
\thispagestyle{fancyOdd}

Видно, что порядок 2 по модулю 13 равен 12, в то время как порядки $2^2$ и $2^3$ равны 6 и 4 соответственно; легко проверить, что
\begin{center}
    $6=12/\text{НОД}(2,12)$ и $4=12/\text{НОД}(3,12)$
\end{center}
в соответствии с Теоремой 8-3. Те числа, которые имеют по модулю 13 порядок 12, являются степенями $2^k$, где $\text{НОД}(k,12)=1$, а именно $$2^5\equiv6,\;2^7\equiv11,\;2^{11}\equiv7\pmod{13}.$$

Если целое число $a$ имеет наибольший возможный порядок, оно называется примитивным корнем $n$.

\begin{definition}
	Если $\text{НОД}(a,n)=1$ и порядок $a$ по модулю $n$ равен $\phi(n)$, тогда $a$ -- \emph{примитивный корень} $n.$
\end{definition}

Другими словами, $n$ имеет примитивный корень $a$, если $a^{\phi(n)}\equiv1\pmod{n}$, но $a^k\not\equiv1\pmod{n}$ для любого положительного $k<\phi(n)$.

Легко увидеть, что 3 -- это примитивный корень 7 так как $$3^1\equiv3,\; 3^2\equiv2,\; 3^3\equiv6,\; 3^4\equiv4,\; 3^5\equiv5,\; 3^6\equiv1\pmod{7}.$$

В более общем случае можно доказать, что примитивные корни существуют для любого простого модуля, что является результатом фундаментальной важности. Несмотря на то, что примитивный корень $n$ может существовать, когда $n$ -- составное (например, 2 -- это примитивный корень 9), не каждое целое число будет обладать примитивным корнем; существование примитивного корня -- это больше исключение, чем правило.\\

\textbf{Пример 8-3}
\begin{quote}
	Покажем, что если $F_n=2^{2^n}+1, \;n>1$ -- простое, то 2 не является примитивным корнем $F_n$. (Очевидно, что 2 -- примитивный корень $5=F_1$.)
	Так как $2^{2^n+1}-1=(2^{2^n}+1)(2^{2^n}-1)$, имеем $$2^{2^{n+1}}\equiv1\pmod{F_n},$$ что значит, что порядок 2 по модулю $F_n$ не превосходит $2^{n+1}$. Но если предполагается, что $F_n$ -- простое, то $$\phi(F_n)=F_n-1=2^{2^n}$$ и индуктивно можно подтвердить, что $2^{2^n}>2^{n+1}$, когда $n>1$. Таким образом порядок числа 2 по модулю $F_n$ меньше, чем $\phi(F_n)$; ссылаясь на определение 8-2, увидим, что 2 не может быть примитивным корнем $F_n$.
\end{quote}
\end{document}
